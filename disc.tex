\section{Lessons learnt}\label{sec:disc}

Power management is still a challenge but, based on the evolution of tools and interfaces that we've explored 
over the course of working on this project, the tools needed to create a more effecient power management system 
are improving all the time. With RAPL (on Intel's processors at least) and libraries like PAPI, accurate power 
estimation is now much simpler. Our implementation can be extended to work with real time, interactive applications 
by using the CPU's performance counters with RAPL's power estimates.

\paragraph{}Speaking of interactive applications, it became apparent to us that perfromance benchmarks might not be the best method  
of evaluating a power management policy, based on our tests. The performance centric nature of benchmarks like SPEC 
are good workloads for power management policies and, as can be seen from our comparisons, might not be the best way 
of differentiating any potential improvements,

\paragraph{} Looking closer at the results of our Linux comparisons, and taking into consideration the possible issues 
with using performance benchmarks, it would seem like the P-state drivers do not make power management a solved issue. 
Based on our analysis, there are gains to be had with application specific policies.
