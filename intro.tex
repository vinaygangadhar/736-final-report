\section{Introduction}\label{sec:intro}\label{sec:motiv}

The last decade of computing has seen mobile phones, wearables and other battery powered systems becoming 
more ubiquitous. This shift towards mobile computing has lead to system designers being faced with the  
challenge of prolonging battery life by limiting the energy consumed  
by system resources (CPU, Memory, Network) and improving system energy efficiency ~\cite{martin-bat, worldwide} while running a multitude of user applications. 
In addition to energy consumption, managing system power dissipation is also very 
important in avoiding heat related problems that are prevalent in computing systems~\cite{hybdtm}.

Most of the existing work on system energy conservation rely on techniques that  
dynamically characterize power consumption and the Quality of Service (QoS) requirements 
of applications. Based on these parameters, systems implement power management policies by 
using DVFS (Dynamic voltage frequency scaling)~\cite{dvfs} to reduce the CPU operating frequency or switching to lower-CPU power states~\cite{sleepscale, ecos} during periods of low activity.
Reflecting this, the current power management governors in Linux primarily use one source of 
information while regulating power, the CPU usage and system performance requirements. 
Based on the performance needs of a system, these governors vary the CPU frequency in an effort to manage energy consumption while maintaining QoS. 
There exists a wide variety of other source of information, through platform independent interfaces like ACPI~\cite{acpi, freqgov}, platform dependent interfaces like RAPL 
and the user as well. The rigid nature of current power management policies could harm the energy efficiency of a system since applications could differ in their needs, and thus 
require different strategies to achieve efficient power usage. 
Some applications (such as those that interact with the user) run for short periods of time and require an immediate response. Other, more background tasks (like a disk scan) run for longer durations 
but might not be as latency bound.
While the former class of applications could benefit from short periods of high performance, the latter could be executed more efficiently by operating at a lower performance level. 
Furthermore, similar issues occur with other system resources and the efficient management of these resources also affect the optimal power management strategy. 
These issues could affect systems that only manage power through CPU frequency scaling as well.

Given the issues identified in the preceding paragraph, we aim to explore cases where the current power governors on Linux fail to yield a reasonable energy management solution. 
Furthermore, similar to the work of Liang et. al~\cite{and-dvfs} we aim to show that by providing information that characterizes the needs of a particular application, 
better energy efficiency is achievable. We first profile applications and characterize them into buckets based on how compute intensive, cache sensitive and memory intensive they are. 
Based on this information, an analytical model called E-MOS is used to reason about the frequency scaling of cores and achieve better energy efficiency. 
The analytical model uses a decision table which is fed the parameters of an application and power management objectives to be achieved as inputs.
It then computes possible actions that need to be taken to optimize energy efficiency.
The model achieves a tradeoff between energy and performance by reducing, increasing or maintaining the CPU and DRAM frequency. 
We evaluate the model's results by implementing its suggestions and testing with the Userspace power governor for a variety of applications. Our experiments only scale CPU frequency but we 
observe X\% of energy savings with Y\% of performance loss compared to the default Linux power governors.

This paper makes the following contributions:
\begin{itemize}
\item  An analysis and case study of existing Linux power governors for different applications that can be categorized as compute intensive, cache sensitive and memory bound. 
\item  An application aware analytical model called E-MOS which analyzes the tradeoff between energy efficiency and performance with application information.
\item  An improved energy-management policy, which tunes the CPU frequency to meet application demand while achieving better energy efficiency.
\item  A comparison of E-MOS with Linux's default power governors for variety of benchmarks.
\end{itemize}

\paragraph{Paper Organization} 
We first discuss related work in the power management field in Section ($\S$\ref{sec:rel}). 
We present a brief overview of the existing Linux power management solutions in Section ($\S$\ref{sec:linux-powergov}). 
Section ($\S$\ref{sec:case-study}) discusses a case-study of non-linear scaling with existing power governors and motivates our work.
Section ($\S$\ref{sec:appl}) explains the application categorization and reason out the 
problems with existing Linux governors. 
Section ($\S$\ref{sec:analytic}) explains
our E-MOS analytical model and the decision table it uses to make application-aware energy management policy decisions. Section ($\S$\ref{sec:meth}) details our experimental methodology 
while Section ($\S$\ref{sec:results}) gives a detailed evaluation and results of our model. We discuss the
lessons we learned in Section ($\S$\ref{sec:disc}) and finally conclude in Section ($\S$\ref{sec:conc}).

