\section{Motivation}\label{sec:motiv}

Modern multi-core computing systems (Laptops and hand-held devices) have major challenge of limiting the energy consumed by the system resources (CPU, Memory, Network) utilized when running applications. Most of the work on energy conservation in operating systems mainly rely on techniques which dynamically characterize power consumption and Quality of Service (QOS) of applications and provide a policy for users by using DVFS (Dynamic voltage frequency scaling)~\cite{dvfs} or lower-CPU power states~\cite{sleepscale, ecos}. 

Linux OS provide ACPI~\cite{acpi, freqgov} interfaces to extract information about batteries and other resources, but the power governors don't utilize this information efficiently to make proper energy management decisions. Current policies generally rely on user-specified static parameters to run at a specific lower CPU frequency or lower system power state without considering the applications’ requirements. Many multi-core processors try to improve the applications execution time by running at higher frequencies and thus consume more power (May consume less energy, if the execution time is small). Some applications have more memory accesses, in which case running at higher frequencies always is not beneficial. Many applications are long running jobs which consume lot of energy if run at higher frequency. All these above scenarios suggest that there is ‘NO’ one-size fit all solution for better system energy efficiency. Thus, Linux power governors need more application information, to determine whether to boost the performance or run at lower-power state by continuously monitoring the available battery resource. Our project aims to solve this problem of energy management decision by deriving some of the principles from Liang et. al~\cite{and-dvfs} implemented for Android systems.  


