\begin{abstract} \vspace{0.05in}

Current Linux Operating Systems' power governors do not provide a fine-grained control and management 
over the energy utilized by the applications running on a computing system. 
Linux power governors by default use "Ondemand" CPU frequency governor for 
power management by considering the entire system load rather than individual application requirements. 
This system-wide and default kernel policy could have undesirable effects of OS power management 
and could lead to poor battery life and performance, 
atleast for modern CPUs.
%Linux power governors operate on a generic energy management policy,
%based on the assumption that reducing frequency of CPU (using DVFS) or 
%putting the system to low-power state would reduce the system energy always,
%without having necessary information about the application. 
In this project, we provide an overview of the difficulties of Linux OS power management system and 
reason out the anomalies with different existing Linux power governors.

We present E-MOS (Efficient Energy Management Policies in Operating Systems),
a model which uses pre-characterized information of applications, to 
make decisions of frequency scaling to achieve 
better performance and energy efficiency. An application aware policy
can be applied using Linux User-space power governors, to
tradeoff performance and energy consumption. E-MOS is
evaluated on variety of benchmarks and we see energy savings 
upto X\% for a Y\% performance loss.




\end{abstract}

