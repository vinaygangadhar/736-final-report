\begin{abstract} \vspace{0.05in}

Power management in Linux is either handled by static, fixed policy power governors or Intel's P-state drivers. 
The former was designed to manage CPU idle states and scales frequency to do so while the latter was designed 
with the race-to-idle (or race-to-sleep) concept. Both of these solutions manage system power by changing the 
performance state of the CPU but they take a more holistic view of the system that could miss some of the nuances  
present in the active applications.

%Linux power governors operate on a generic energy management policy,
%based on the assumption that reducing frequency of CPU (using DVFS) or 
%putting the system to low-power state would reduce the system energy always,
%without having necessary information about the application. 
In this project, we explore the opportunities that application characterization presents for more efficient power management 
and provide an overview of the difficulties in Linux power management. 

We present E-MOS (Efficient Energy Management Policies in Operating Systems),
a power management model which uses applications characteristics (CPU/memory bound and cache sensitivity) to 
make frequency scaling decisions in order to achieve better energy efficiency while balancing the performance and energy requirements of a system. 
The decisions of this application aware policy 
can be applied through the ACPI (Advanced Configuration and Power Interface) userspace power governor. 
E-MOS is evaluated on variety of benchmarks and we see energy savings 
up to 2x for a 13\% performance loss.




\end{abstract}

\begin{comment}
Current Linux Operating Systems' power governors do not provide a fine-grained control and management 
over the energy utilized by the applications running on a computing system. 
Linux power governors by default use "Ondemand" CPU frequency governor for 
power management by considering the entire system load rather than individual application requirements. 
This system-wide and default kernel policy could have undesirable effects of OS power management 
and could lead to poor battery life and performance, 
at least for modern CPUs.
%Linux power governors operate on a generic energy management policy,
%based on the assumption that reducing frequency of CPU (using DVFS) or 
%putting the system to low-power state would reduce the system energy always,
%without having necessary information about the application. 
In this project, we provide an overview of the difficulties of Linux OS power management system and 
reason out the anomalies with different existing Linux power governors.

We present E-MOS (Efficient Energy Management Policies in Operating Systems),
a model which uses pre-characterized information of applications, to 
make decisions of frequency scaling to achieve 
better performance and energy efficiency. An application aware policy
can be applied using Linux User-space power governors, to
trade off performance and energy consumption. E-MOS is
evaluated on variety of benchmarks and we see energy savings 
up to X\% for a Y\% performance loss.

\end{comment}
