\section{Related Work}\label{sec:rel}

The Energy Management (EM) area in operating systems constitutes a diverse set of solutions to the problem of ensuring performance while decreasing power usage. 
The Linux kernel has some predefined governors for managing power but many papers have recognized that these general purpose solutions are certainly not optimal, 
and perhaps in some case not even adequate. 
There have been a number of attempts to remedy this and we enlist some of these solutions in this section. Dynamic Voltage Frequency Scaling~\cite{dvfs} has been a 
popular hardware mechanism to save power and boost performance. Zeng et. al~\cite{ecos} focuses on directly managing power usage as one might manage other scarce resources. 
Choi et. al~\cite{decomp} dynamically profile applications to determine a good energy management policy at hardware level. 

\paragraph{Non-Optimal General Purpose EM Policies} It has been recognized that the generic policies for energy management 
implemented in Linux by using the ACPI interface are not always the best performing, or most efficient. For example in the 
context of mobile devices~\cite{and-dvfs}, the on-demand power governor used by the Android operating system does not guarantee 
low power consumption for all workloads. Recently (as of linux 3.9), Intel introduced a P-state driver that is more platform specific 
to overcome some of the issues of the ACPI governors. On the opposite end of the spectrum, we see work done by Yanpei. et. al~\cite{sleepscale} that 
attempts to tackle this issue for datacenter workloads and emphasizes QOS guarantees rather than energy efficiency. Our work focuses on 
providing more dynamic energy management decisions based on the application information. Our work is similar to the work 
in ~\cite{and-dvfs}, in the sense that we use the UserSpace power governor to drive our power policy and compare our implementation with 
the other Linux governors. 

\paragraph{Existing Energy Efficiency Policies} In the case of~\cite{ecos}, energy is treated as another scarce resource and a currency based algorithm is used. 
Processes in this system are allocated a specific amount of power usage, and are given access to various power consuming operations in relation to the amount 
of power they are allocated. However, this is focused largely on the question of optimization of allocation, and not on the question of how to ensure that the energy is productively used.
Other implementations~\cite{and-dvfs, decomp} attempt to form a correlation between the number of memory accesses and ideal operating frequency. They base this 
argument on the fact that some workloads see lower power when executed at higher frequencies. Their experiments show that this is a consequence of the 
number of memory accesses, whose frequency of operation remains constant. By getting a measure of the memory accesses in a workload, 
they use the memory access - CPU frequency correlation to set the operating frequency.
We have a similar approach where we classify workloads depending on how CPU/memory bound and cache sensitive. 
We use this information to make power management decisions for these classified workloads. 

\paragraph{Application Profiling and decomposition} Dynamic voltage and frequency scaling based on application profiling and 
decomposition has been done in Choi. et. al~\cite{decomp}. This idea on profiling is similar to our project’s initial profiling step, but the workload 
is decomposed into two-parts here: On-chip and Off-chip, effectively indicating CPU sensitive and memory sensitive applications respectively. 
It exploits the idea that different workloads have different power management needs and these statistics could be exploited 
at run-time. They do not implement this in operating system and the energy management policy  
may be overriden by the kernel's own policies. Similarly, Choi et. al~\cite{choi2005fine} work does DVFS for energy conservation 
classifying all the applications based on On-chip and Off-Chip computation ratio. Again this would result in a generic decision, based on ratios calculated offline. 
Power conscious fixed scheduling policy of applications with DVFS based on profiling information has been implemented in Shin et. al~\cite{shin1999power}, 
which again does not integrate the decision making capability to operating systems. 
Snowdon et. al~\cite{snowdon2009koala} also characterize applications but use this information to estimate an application's power consumption. 
They use this information to present the OS with a flexible power management policy and, unlike the other work in this paragraph, implement Koala in Linux.
Similar to Koala, our project aims at embedding application information into the energy management policy of the operating system, based on an application's needs. 
